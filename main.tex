%%%%%%%%%%%%%%%%%%%%%%%%%%%%%%%%%%%%%%%%%%%%%%%%%%%%%%%%%%%%%%%%%%%%%%%%%%%%%%%%%%%
%%%  Filename: main.tex
%%%  ---
%%%  Template for Doctoral Dissertation at 
%%%  Doctoral Program in Computer Science, DIKE FMIPA UGM   		
%%%  Created using disertasidikeugm.cls
%%%  --- 
%%%  Developed by Hanif Rahmat
%%%	 Doctoral Program in Computer Science, DIKE FMIPA UGM
%%%  [hanifrahmat123@gmail.com]
%%%  Github repo: https://github.com/hanifrahmath
%%%  ---
%%%  Thanks to Canggih Puspo Wibowo [canggihpw@gmail.com] who wrote the original version of the template, and 
%%%  Dr.-Ing. Yohan Fajar Sidik [DTETI FT UGM] [yohanfajarsidik@ugm.ac.id] who further developed the template.
%%%  Github repo: https://github.com/Dr-Sidik/template_thesis_latex_dteti
%%%%%%%%%%%%%%%%%%%%%%%%%%%%%%%%%%%%%%%%%%%%%%%%%%%%%%%%%%%%%%%%%%%%%%%%%%%%%%%%%%%

%% Use option "bahasa" or "english" 
%%    to change the basic language used
% \documentclass[<bahasa/english>]{disertasidikeugm}
\documentclass[bahasa]{disertasidikeugm}

%======================================
% Information Input
%======================================
% Input author's name and ID number
\author{<<Nama Mahasiswa>>}{<<NIM>>}
% Input the thesis' title
\title{<<Judul Disertasi Bahasa Indonesia>>}
\titleeng{\textit{<<Judul Disertasi Bahasa Inggris>>}}
\yearsubmit{2026}
% Name of thesis supervisors/promotors
\addsupervisor{<<Promotor>>}{NIP. <<NIP>>}
\addsupervisor{<<Ko-promotor>>}{NIP. <<NIP>>}
%\addsupervisor{<<Supervisor 3>>}{<<NIP>>}
% Name of examiners
%\addexaminer{<<Examiner 1>>}{<<NIP 1>>}
%\addexaminer{<<Examiner 2>>}{<<NIP 2>>}
%\addexaminer{<<Examiner 3>>}{<<NIP 3>>}
%\addexaminer{<<Examiner 4>>}{<<NIP 4>>}
%\addexaminer{<<Examiner 5>>}{<<NIP 5>>}
%\addexaminer{<<Examiner 6>>}{<<NIP 6>>}
%\addexaminer{<<Examiner 7>>}{<<NIP 7>>}
%\addexaminer{<<Examiner 8>>}{<<NIP 8>>}
%\addexaminer{<<Examiner 9>>}{<<NIP 9>>}

%======================================

% correct bad hyphenation here [example]
% \babelhyphenation[<<english/bahasa>>]{op-tical net-works semi-conduc-tor}
%% Uncomment block of code below to disable hyphenation
\tolerance=1
\emergencystretch=\maxdimen
\hyphenpenalty=10000
\hbadness=10000

\begin{document}
%======================================
% Create cover etc
%======================================

%---- COVER ----
\printcover{sample/logougm.png}

%---- ENDORSEMENT PAGE ----
% Select endorsement page type. If you want to use your own PDF file,  
% 	use \printendorsementpdf, or if you want to use JPG file, use 
% 	\printendorsementjpg. Otherwise, use \printendorsement.
% 	Choose one only. Comment out unused command(s).
%
\cleardoublepage \phantomsection
\printendorsement
%\printendorsementpdf
%\printendorsementjpg{sample/scanned-endorsement.jpg}

%---- STATEMENT PAGE ----
% Select statement page type. If you want to use your own JPG file,  
%	use \chapterstatementjpg{<your *.jpg file path>}. Otherwise, 
%	use \chapterstatement{contents/statement/statement}.
%	Choose one only. Comment out unused command(s).
\cleardoublepage \phantomsection
\chapterstatement{contents/statement/statement}
%\chapterstatementjpg{sample/scanned-statement.jpg}

%---- DEDICATION PAGE ----
\cleardoublepage \phantomsection
\chapterdedication{contents/dedication/dedication}

%---- PREFACE PAGE ----
\cleardoublepage \phantomsection
\chapterpreface{contents/preface/preface}

%======================================
% Create Table of Contents, List of Figures, List of Tables
% <Do not change this part>
%======================================
\cleardoublepage \phantomsection
\thetoc
\onehalfspacing
\tableofcontents
\singlespacing
\cleardoublepage \phantomsection
\thelot
\listoftables
\cleardoublepage \phantomsection
\thelof
\listoffigures

%======================================

%---- NOMENCLATURE PAGE ----
%\cleardoublepage \phantomsection
%\chapternomenclature{contents/nomenclature/nomenclature}

%---- INTISARI PAGE----
\cleardoublepage \phantomsection
\chapterintisari{contents/abstract/intisari}

%---- ABSTRACT PAGE----
\cleardoublepage \phantomsection
\chapterabstract{contents/abstract/abstract}

%======================================

%======================================
%  MAIN TEXT
%======================================
\startmain
% You can change 
%    the filename and location of the files inputted
\cleardoublepage \phantomsection
\chapter{PENDAHULUAN}

\section{Latar Belakang}

Ini adalah latar belakang.

\section{Rumusan Masalah}

Ini adalah rumusan masalah.

\section{Batasan Penelitian}

Ini adalah batasan penelitian.

\section{Tujuan Penelitian}

Ini adalah tujuan penelitian.

\section{Manfaat Penelitian}

Ini adalah manfaat penelitian, di antaranya:

\begin{enumerate}
	\item Manfaat 1.
	\item Manfaat 2.
	\item Manfaat 3.
\end{enumerate}

\section{Kontribusi Penelitian}

Ini adalah kontribusi penelitian.



\cleardoublepage \phantomsection
\chapter{KAJIAN PUSTAKA}

Ini adalah kajian pustaka.


\cleardoublepage \phantomsection
\chapter{LANDASAN TEORI}
\label{chap:landasan-teori}

Ini adalah landasan teori.

\section{Subbab 1}
\label{sec:subbab1}

Persamaan \ref{eq:contoh-persamaan1}. Persamaan \ref{eq:contoh-persamaan2}. Gambar \ref{fig:contoh-gambar}. Tabel \ref{tab:contoh-tabel}. \citet{Vaswani2017} membahas Transformer \citep{Vaswani2017}. \citet{Wang2022} membahas metode \textit{object detection} pada skenario menyetir berbasis YOLOv4 termodifikasi \citep{Wang2022}. Lihat Bab \ref{chap:landasan-teori}. Lihat Subbab \ref{sec:subbab1}.

\begin{equation} 
	\label{eq:contoh-persamaan1}
	\begin{split}
		\mathrm{SA}(X) &= \mathrm{Softmax}\left( \frac{QK^T}{\sqrt{d_k}} \right)V, \\
		\text{di mana} \quad Q &= XW_q, K=XW_k, V=XW_v
	\end{split}
\end{equation}

\begin{equation}
	\label{eq:contoh-persamaan2}
	\mathcal{L}_{\mathrm{box}}(b_i, \hat{b}_{\sigma(i)}) = \lambda_{\mathrm{iou}} \mathcal{L}_{\mathrm{iou}}(b_i, \hat{b}_{\sigma(i)}) + \lambda_{\mathrm{L1}} \, \lVert b_i - \hat{b}_{\sigma(i)} \rVert
\end{equation}

\begin{figure}[H]
	\centering
	\includegraphics[width=0.5\linewidth]{contents/chapter-3/contoh-gambar.jpg} % mendukung format PDF, JPG, PNG
	\caption{Contoh gambar}
	\label{fig:contoh-gambar}
\end{figure}

\begin{table}[H]
	\centering
	\caption{Contoh tabel}
	\label{tab:contoh-tabel}
	\begin{tabular}{ll}
		\hline
		Kata & Vektor representasi \\
		\hline
		\texttt{<start>}   & [0,46; 0,24; 0,55; 0,12] \\
		\texttt{Saya}      & [0,45; 0,67; 0,90; 0,14] \\
		\texttt{adalah}    & [0,34; 0,33; 0,57; 0,55] \\
		\texttt{seorang}   & [0,56; 0,87; 0,34; 0,67] \\
		\texttt{mahasiswa} & [0,25; 0,36; 0,54; 0,78] \\
		\texttt{<end>}     & [0,79; 0,44; 0,46; 0,34] \\
		\hline
	\end{tabular}
\end{table}


\cleardoublepage \phantomsection
\chapter{METODOLOGI PENELITIAN}

Ini adalah metodologi penelitian.
\cleardoublepage \phantomsection
\chapter{HASIL DAN PEMBAHASAN}

Ini adalah hasil dan pembahasan.


\cleardoublepage \phantomsection
\chapter{KESIMPULAN DAN SARAN}

\section{Kesimpulan}

Ini adalah kesimpulan.

\section{Saran}

Ini adalah saran.

%======================================

%======================================
%  References
%======================================
\cleardoublepage \phantomsection
\thereferences
% You can change 
%    the filename and location of the files inputted
\bibliography{references}

%======================================

%======================================
%  Appendix
%======================================
% You can change 
%    the filename and location of the files inputted
%    use \chapterappendix for the first page of the appendix
%    use \chapterappendixadd for the next page

%\appendix
%\appendixtables
%\appendixfigures

%\cleardoublepage \phantomsection
%\chapterappendix{contents/appendix/appendix-isi-lampiran}
%\chapterappendixadd{contents/appendix/appendix-latex}
%\chapterappendixadd{contents/appendix/appendix-penulisan-referensi}
%\chapterappendixadd{contents/appendix/appendix-code}




%======================================

\end{document}